%% Elemento obrigatório (Figura 15).
%% Consiste em uma tradução do resumo em português para uma
%% língua estrangeira (em inglês, ABSTRACT; em espanhol, RESUMEN;
%% em francês, RÉSUMÉ), em um único parágrafo, seguido das palavras-
%% -chave representativas do conteúdo do trabalho, na língua estrangeira
%% escolhida.
%% O resumo em outra língua também é precedido pela referência
%% do trabalho, substituindo-se o título em português pelo título na língua
%% estrangeira adotada.
%% No caso de teses, é possível incluir dois resumos em língua es-
%% trangeira.
%% A apresentação gráfica e a ordem dos elementos seguem a mes-
%% ma orientação do resumo em português.

\begin{resumo}[Abstract]
\begin{otherlanguage*}{english}
\begin{SingleSpace}

\noindent
\begin{flushleft}
\entradaAutor{}. \textit{\englishTitle{}} 2023.\pageref{LastPage} f. Trabalho de Conclusão de Curso (Graduação em Engenharia de Computação) - Instituto Politécnico, Universidade do Estado do Rio de Janeiro, Nova Friburgo, 2023.
\end{flushleft}
\vspace{\onelineskip}

\setlength{\parindent}{1.3cm}

The aging population in Brazil poses health challenges, including an increase in Alzheimer's cases and the risk of fractures from falls, impacting the mortality rate of the elderly. Detecting epileptic seizures is crucial due to patients' lack of awareness, complicating the proper adjustment of medication. Identifying incidents such as falls, epileptic attacks, and the user's physical parameters is vital for treatment adjustments. Additionally, user location is essential. In this context, aiming to reduce the consequences of falls, particularly in the elderly, a monitoring and medical assistance system becomes crucial to address these challenges and improve the population's quality of life. In this project, a prototype was developed, creating a real-time monitoring system for patients. The system sends alerts to trusted individuals about falls, the patient's status (conscious and immobile or unconscious and immobile), possible epileptic seizures, and assists in locating patients with Alzheimer's who may be lost. Moreover, the prototype measures the patient's physical parameters and alerts about potential abnormalities. To achieve these objectives, this project utilized the ESP32 microcontroller, coupled with an accelerometer, pulse oximeter, heart rate monitor, and a GPS module. The assembly software was developed within the Arduino IDE platform, while the setup part was implemented in the PHP language. The prototype was validated, and the overall project results were satisfactory, with some improvements needed for blood pressure and body temperature measurements.

\vspace{\onelineskip}
\noindent Keywords: medical care; bracelet; ESP32; health; XAMPP.

\end{SingleSpace}
\end{otherlanguage*}
\end{resumo}